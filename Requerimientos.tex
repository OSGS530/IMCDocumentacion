\documentclass[12pt,a4paper]{article}
\usepackage[utf8]{inputenc}
\usepackage[spanish]{babel}
\usepackage{amsmath}
\usepackage{amsfonts}
\usepackage{amssymb}
\usepackage{graphicx}
\usepackage{array,tabularx}
\usepackage[left=2cm,right=2cm,top=2cm,bottom=2cm]{geometry}
\begin{document}
\title{Documento de requerimientos} 
\include{caratulaTriesgos}
 
\vspace{1 cm}
\section{Introduccion}
	En este documento se hablará acerca de los requerimientos necesarios para la creación de una calculadora para el indice de masa corporal. En la que estaremos involucrando las edades entre 10 y 19 años para tener una mejor comparativa entre las tablas de imc, ya que apartir de los 20 años tanto hombres como mujeres comparten los mismos datos de la tabla.\\
\subsection{Propósito}
Los Riesgos dentro de un proyecto de software son eventos o condiciones inciertas que pueden ocurrir y tener un impacto positivo o negativo sobre los objetivos fijados dentro del proyecto.\\
Es por eso que este documento se realiza con el fin de poder identificar los riesgos de manera clara y así poder atacarlos o prevenirlos con anticipación.\\
\subsection{Visión general de los posibles riesgos}
\textbf {Los riesgos principales tienen que ver con las siguientes categorías:} \\
\begin{itemize}
\item \textit{Problemas del equipo de trabajo.}
\item \textit{Problemas de planificación del proyecto.}
\item \textit{Problemas con el manejo de las tecnologías.}
\item \textit{Problemas con la base de datos.}
\end{itemize}

\subsection{Responsabilidades}
\textbf {Sera responsabilidad del líder de proyecto:} \\
\begin{itemize}
\item \textit{Gestionar el presente plan.}
\item \textit{Gestionar la calidad del proyecto.}
\item \textit{Controlar que las actividades se realicen de acuerdo al plan definido.}
\item \textit{Tomar las medidas necesarias para asegurar que el proyecto se realice en el tiempo establecido.}
\item \textit{Asegurarse que el proyecto cumpla con el objetivo principal.}
\end{itemize}
\vspace{1 cm}

\textbf {Equipo de trabajo:} \vspace{5 mm} \\
Todo el equipo de trabajo deberá tener en cuenta también lo descrito en el presente documento en pos de
disminuir los riesgos del proyecto y así garantizar que se cumplan los objetivos y requerimientos establecidos.
\newpage

\subsection{Tabla de riesgos del proyecto}
\vspace{1 cm}
\textbf{Identificación del riesgo:}
\begin{table}[h!]
\centering
\begin{tabular}{|c|p{8cm}|c|}
\hline
\textbf{ID}&\textbf{Riesgo}&\textbf{Categoría}
\\\hline
R1&Dificultad en la comunicación vía remota entre los miembros del equipo&Equipo de trabajo.\\\hline
R2&No contar con la cantidad de información requerida para el proyecto&Proyecto, Base de datos.\\\hline
R3&El desarrollo y la administración del proyecto lleva más tiempo de lo esperado.&Proyecto, Planificación.\\\hline
R4&Dificultades durante el desarrollo de alguna etapa del proyecto.&Equipo de trabajo.\\\hline
R5&Dificultades con el manejo del repositorio y control de versiones.&Equipo de trabajo, Manejo de tecnologías.\\\hline
R6&Rupturas dentro del repositorio.&Equipo de trabajo, Manejo de las Tecnologías.\\\hline
R7&El proyecto no cumple con los requerimientos establecidos&proyecto, planificación.\\\hline
R8&El proyecto no pasa las pruebas establecidas&proyecto, planificación, equipo de trabajo.\\\hline
\end{tabular}
\end{table}

\subsection{Probabilidad e Impacto}
En el presente análisis se empleará una escala de medición subjetiva expresada
en la siguiente tabla: 
\begin{table}[h!]
\begin{tabular}{|c|p{8cm}|}
\hline
\textbf{Rango de probabilidad}&\textbf{Impacto}
\\\hline
0.10-0.30&Bajo.\\\hline
0.40-0.60&Medio.\\\hline
0.70-100&Crítico.\\\hline
\end{tabular}
\end{table}

En la siguiente tabla se expresan los riesgos identificados para el proyecto con las
probabilidades e impactos estimados subjetivamente para cada uno de ellos: 
\vspace{1 cm}
\begin{table}[h!]
\begin{tabular}{|c|c|p{8cm}|}
\hline
\textbf{Riesgo}&\textbf{Probabilidad}&\textbf{Impacto}
\\\hline
R1&0.40&Medio.\\\hline
R2&0.70&Crítico.\\\hline
R3&0.40&Medio.\\\hline
R4&0.60&Medio.\\\hline
R4&0.30&Bajo.\\\hline
R5&0.40&Medio.\\\hline
R6&0.30&Bajo.\\\hline
R7&0.30&Bajo.\\\hline
\end{tabular}
\end{table}

\subsection{Supervisión y gestión del riesgo}
Además de haber identificado y listado los posibles riesgos es necesario que este documento sea revisado frecuentemente por todos los involucrados en el proyecto, ya que es posible identificar nuevos riesgos a lo largo del desarrollo de este proyecto y tendrá que ser actualizado para hacer una correcta gestión de los riesgos.
\vspace{1 cm}
\begin{table}[h!]
\begin{tabular}{|p{0.10\linewidth}|p{0.30\linewidth}|p{0.40\linewidth}|p{0.15\linewidth}|}
\hline
\textbf{ID Riesgo}&\textbf{Importancia del riesgo}&\textbf{Cómo minimizarlo}&\textbf{Quien lo supervisa}
\\\hline
R1&Porque esta dificultad por parte de los integrantes del equipo de trabajo pondría en riesgo el desarrollo en tiempo y forma del proyecto.&\begin{itemize}
\item \textit{Establecer reuniones donde los integrantes puedan estar en constante comunicación.}
\item \textit{Darle seguimiento a las tareas que cada uno desempeña.}
\end{itemize}
&Líder del proyecto.\\\hline

R2&Porque se tendría que cambiar el tema del proyecto a desarrollar.&\begin{itemize}
\item \textit{Hallar la base de datos Spotify que contenga al menos 1 millón de registros.}
\item \textit{Proponer alternativas eficientes en caso de no hallar la información exacta.}
\end{itemize}&Equipo de trabajo, Líder del proyecto.\\\hline

R3&Porque en caso de retrasarse el proyecto se vería afectada la entrega final del proyecto y la evaluación parcial de los integrantes del proyecto.&\begin{itemize}
\item \textit{Revisar que las actividades de cada integrante del equipo de trabajo se estén realizando de acuerdo a lo establecido.}
\item \textit{Utilizar kanban automatizados con reviewers que estén monitoreando que se lleven a cabo las actividades.}
\end{itemize} 
&Líder del proyecto.\\\hline

R4&Porque esta dificultad pone en riesgo la entrega final del proyecto y un posible retraso en las etapas siguientes.&\begin{itemize}
\item \textit{Estar continuamente revisando que los Issues se estén realizando.}
\item \textit{Solicitar apoyo al líder del proyecto con las posibles dudas o imprevistos.}
\end{itemize}
&Líder del proyecto, Equipo de trabajo.\\\hline
\end{tabular}
\end{table}

\newpage

\begin{table}[h!]
\begin{tabular}{|p{0.10\linewidth}|p{0.30\linewidth}|p{0.40\linewidth}|p{0.15\linewidth}|}
\hline
\textbf{ID Riesgo}&\textbf{Importancia del riesgo}&\textbf{Cómo minimizarlo}&\textbf{Quien lo supervisa}
\\\hline

R5&Porque es necesario llevar el control de los commits de los avances del proyecto y poder sacar un release para hacer las pruebas correspondientes.&\begin{itemize}
\item \textit{Asignar a un integrantes del equipo que tenga un manejo de git.}
\item \textit{Estar en constante comunicación con el líder del proyecto para que apruebe los cambios e integraciones nuevas de código.}
\end{itemize}&Equipo de trabajo, Líder del proyecto.\\\hline

R6&Podría repercutir gravemente en el avance del proyecto causando problemas en la funcionalidad del código ya en producción o en alguna otra etapa importante de este.&\begin{itemize}
\item \textit{Se deberá solicitar Pull Request al líder del proyecto para la integración de las nuevas funcionalidades del proyecto y revisión de las mismas.}
\item \textit{Habilitar la protección de cada Branch a través de la integración continua definiendo las reglas necesarias en cada Branch.}
\end{itemize} 
&Líder del proyecto.\\\hline

R7&Porque no se cumpliría con el objetivo principal, dejando de lado los requerimientos del sistema.&\begin{itemize}
\item \textit{Llevar un buen control de la documentación y planificación del proyecto.}
\item \textit{Hacer revisiones continuas de cada avance del proyecto.}
\end{itemize} 
&Equipo de trabajo, Líder del proyecto.\\\hline

R8&Porque no se podría mandar esa parte del desarrollo a producción, causaría retrasos y no cumpliría con los requerimientos del sistema .&\begin{itemize}
\item \textit{Codificar las pruebas necesarias para hacer cumplir los requerimientos.}
\item \textit{Se deberá volver a la etapa de desarrollo para hacer las correcciones necesarias y así garantizar que se pasen todas las pruebas.}
\end{itemize} 
&Equipo de trabajo, Líder del proyecto.\\\hline

\end{tabular}
\end{table}

\end{document}
