\documentclass[12pt,a4paper]{article}
\usepackage[utf8]{inputenc}
\usepackage[spanish]{babel}
\usepackage{amsmath}
\usepackage{amsfonts}
\usepackage{amssymb}
\usepackage{graphicx}
\usepackage{array,tabularx}
\usepackage[left=2cm,right=2cm,top=2cm,bottom=2cm]{geometry}
\begin{document}
\begin{titlepage}
	\begin{center}
	{\huge \textbf{Universidad Veracruzana}}\\
	\vspace{2cm}  
	{\Large {Documento de requerimientos}}\\
	\vspace{5mm}	
	{\Large {Calculadora IMC}}\\
	\begin{figure}[h]
		\centering
		\includegraphics[scale=0.10]{uvlogo}
	\end{figure}
	{\Large {Ingeniería de software}}\\
    \vspace{2cm}
	{\Large {García Sosa Oswaldo }}\\
	\vspace{5mm}	
	{\Large {Martínez Espinosa Gerardo Iván}}\\
	\vspace{15mm}	
    \rule{8cm}{0.5mm} \\ \Large Vo.bo\\ 
	\end{center}
\end{titlepage}
\vspace{1 cm}
\section{Introduccion}
	En este documento se hablará acerca de los requerimientos necesarios para la creación de una calculadora para el indice de masa corporal. En la que estaremos involucrando las edades entre 10 y 19 años para tener una mejor comparativa entre las tablas de imc, ya que apartir de los 20 años tanto hombres como mujeres comparten los mismos datos de la tabla.\\
\subsection{Propósito}
	La calculadora de indice de masa coporal v1.0 tiene como proposito dar el imc a cualquier persona que este interesado en saberlo. Sabiendo que es un dato muy importante para la salud, por lo cual es de manera gratuita, intuitiva y facil de consultar. Solo requeriras datos personales, como genero, edad, estatura y peso.\\
\subsection{Alcance}
	El presente documento especifica los requerimientos que se deben tomar en cuenta en las fases posteriores del desarrollo de la calculadora para el indice de masa corporal, contiene los requerimientos funcionales, no funcionales y asi como los roles que cumple cada usuario.\\

\subsection{Referencias}
	Con dichos datos el sistema debe calcular el IMC segun la tabla proporcionada en:\\
		http://www.imss.gob.mx/salud-en-linea/calculaimc\\
	
\subsection{Funcionalidades}
\begin{itemize}
	\item\textit{Calcular el IMC hombre}
	\item\textit{Calcular el IMC mujer}
	\item\textit{Dar resultado de IMC}
	\item\textit{Mostrar de acuerdo a la tabla donde se encuentra}\\
\end{itemize}

\subsection{Entorno operativo}
	La calculadora del imc estara por medio de una pagina web desarrllada por medio de Angular, sera posible acceder a ella por medio de dos maneras. La primera sera atravez del servidor de la clase "Pruebas de Software" y el segundo atraves de un link proporcionado por firebase.\\

\subsection{Requerimientos funcionales}
\begin{itemize}
	\item\textit{El sitema recibe por entrada los siguientes valores:}\\
			Edad: NUmero entero sin parte decimal\\
			Peso: Numero con parte decimal (Opcional) representando el peso de la persona en kilos\\
			Altura: Nunero entero sin decimal representado la altura de una persona en centimetros\\
	\item\textit{Se debe calcular el imc para hombre y mujer entre edades de 10 y 19 años}\\
	\item\textit{Se espera que el sistema tenga como salida informacion para el usuario que contenga:}\\
			Su imc calculado\\
			Su estado de peso segun la tabla proporcionada\\
\end{itemize}

\subsection{Requerimientos de interfaces externas}
	La interfaz de la calculadora sera por medio de una pagina web, en la que se encontraran apartados para poner cada uno de los valores previamente dichos (Edad, genero, peso y estatura), se inclura un boton para la confirmacion de los datos y asi mismo realizara los calculos del imc.\\
	Una vez presionado el boton se dezplegara en la parte inferior de este informacion relacionada con los calculos realizados y asi mismo, dira en que apartado de la tabla del imc se encuentra.\\
	
\subsection{Requerimientos no funcionales}
\begin{itemize}
	\item\textit{El sistema debe funcionar en un navegador web Chrome en su ultima version, Firefox en la 			ultima version, Edge version mayor a las dos ultimas lanzadas, IE 11.}
	\item\textit{El sistema debe dar salida con un tiempo maximo de 2 segundos.}
\end{itemize}

\end{document}
